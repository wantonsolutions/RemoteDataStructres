
\section{Background}
\label{sec:background}


%\subsection{RDMA Key-Value Stores}

%% %Describe existing key value stores relation to RDMA and
%serialization %
Key/value stores have long looked to RDMA to
%Many non-disaggregated key-value stores have used RDMA to
accelerate their
performance~\cite{farm,memc3,erpc,herd,faast,mica,pilaf,cell,storm}.
%%
These systems strike a careful balance between directly accessing hash
tables stored in remote memory with one-sided RDMA requests and
serializing mutating operations at a server-side CPU.
%%
Cuckoo~\cite{cuckoo} and hopscotch~\cite{hopscotch} hashing are popular choices in
this space because clients can independently calculate the location of
keys without needing to consult the server and potentially perform
lock-less reads with RDMA~\cite{farm,pilaf,cuckoo}.



\subsection{RDMA}

RDMA
%is an enabling technology for memory disaggregation. It allows
allows
clients to directly access the memory of a remote server with
one-sided operations like read, write and atomic update without the
involvement of a remote CPU~\cite{infiniband-spec}.  To set the
context for RCuckoo's design, we benchmark the baseline performance of
our testbed hardware, illustrating the constraints and
opportunities.

%%
Atomic operations such as compare-and-swap (CAS) are essential for
implementing locks or opportunistic concurrency. Atomics are limited
to 64-bit operations and bottleneck at lower rates than reads and
writes because they block requests on data-dependent addresses while
waiting on PCIe round trips~\cite{design-guidelines,sherman}.
Figure~\ref{fig:rdma-benchmarks}(a) shows that the NICs in our testbed
(100-Gbps NVIDIA Mellanox ConnectX-5s) are capable of serving many 10s of
millions of RDMA operations per second (limited only by link speed),
but CAS operations to remote server memory top out around 50 MOPS.  Hence, we employ atomic operations judiciously in RCuckoo.

%the scalability of
%these operations on CX5 NICs for 64 bit operations.
%%

While atomic operations are limited to 64 bits, read and write message
sizes are bounded only by the link MTU.
%can 
%RDMA NIC bandwidth has increased disproportionately to
%latency in the last decade leading to interesting design
%tradeoffs. CX7 NICs have 10$\times$ the bandwidth of CX3's
%but their intra-rack RTT has only shrunk by 1.5$\times$.
Figure~\ref{fig:rdma-benchmarks}(b) shows that on our testbed
NIC-to-NIC round-trip times are similar for all message sizes
less than about 128 bytes, and
%on CX5s for variable sized writes. All writes up
%to 128 bytes have comparable latencies and
messages must exceed 1~KB before the latency of a single large operation
exceeds two RTTs of smaller ones.  We leverage this
observation in RCuckoo by collapsing multiple small reads into a single
large one when appropriate.

%\textbf{If there
%is surplus bandwidth a single large message can have much
%lower latency than two dependent smaller messages.}

Finally, we highlight that our testbed NICs, like those from several
vendors, include a small amount (256~KB in our case) of on-NIC memory
that can be addressed by remote clients using RDMA
operations~\cite{device-memory}.  Accesses to NIC memory avoid the
need to cross the server's PCIe bus, decreasing latency and increasing
throughput.  The performance gain is particularly significant for
atomic operations.  Figure~\ref{fig:rdma-benchmarks}(c) shows the
maximum aggregate throughput of concurrent CAS operations targeting
the same single (i.e., contended) address or distinct, independent
addresses in both main server memory (shown in orange) and on-NIC
device memory (blue).  CAS operations perform between 1.8--3.1$\times$
faster on NIC memory.  RCuckoo's datastructures are designed so that
all CAS operations are performed on NIC memory.

%many modern NICs
%Vendors have provided some workarounds for atomic
%bottlenecks by adding device memory and masked atomic
%operations. CX series NICs have a 256KB region of on-NIC
%RDMAable memory. Atomics to this region avoid the round trip
%and execute with lower latency and up to 3x higher
%throughput().
%Masked CAS (MCAS) allows for each bit to be set
%independently thus enabling higher density locks while
%reducing contention~\cite{rdma-masked-cas}. While multi-CAS
%is not supported these features have been demonstrated to
%enable fast dense lock tables~\cite{sherman}.

\subsection{Concurrent hash tables} 
\label{sec:cuckoo-back}

Fully disaggregated key/value stores are essentially concurrent hash
tables whose conflict resolution strategy is implemented entirely by
individual clients~\cite{rolex,fusee,race}.  Like any hash table, the
underlying hashing algorithm must have an approach to managing
collisions. Cuckoo and hopscotch hashing are particularly attractive
in this context, because they both provide the property that the
potential locations of an entry in the table, regardless of
contention or collision, can be deterministically computed by clients
based only upon the key itself.  Moreover, the set of locations is
limited.  Hence, at least in theory, systems built around either
cuckoo or hopscotch hashing hold the potential for single-round-trip
reads.

%\todo{Make sure to describe how a cuckoo path works}

\textbf{Cuckoo hashing} uses independent hash functions to compute two
potential table locations for a key, a primary and a secondary, where
each location corresponds to an associative row of entries.  A key is
always inserted into its primary location.  If that row is full, an
existing key is evicted (or ``cuckooed'') to its secondary row to make
space. If the cuckooed entry's secondary row is also full, the process
iterates (by selecting yet another entry in the secondary row to
cuckoo) until an open location is found. The path of evictions is
known as a~\textit{cuckoo path}.  While insertions can be involved,
reads can always be executed in a single round trip by reading the
rows corresponding to both of a key's locations
simultaneously~\cite{pilaf}.

\textbf{Hopscotch hashing} works in a similar fashion but provides a
slightly different guarantee, namely that keys will be located within
a bounded neighborhood.  (While cuckoo hashing limits the number of
locations in which a key may be stored, it does not provide any
locality guarantees regarding those locations.) It does so by finding
the physically closest empty entry to the desired location and then,
if that location is not within the desired neighborhood, iteratively
moving other entries out of the way to make room for the new key.
The hopscotch process is facilitated by maintaining a per-entry
bitmask of nearby collisions.  As with cuckoo hashing, clients can complete a read in a single round trip time by reading a key's entire
neighborhood at once.

The insert operation is expensive for both approaches, and prior
systems have taken steps to mitigate its cost.  In associative hashes
like cuckoo hash tables, multiple entries can be chosen as eviction
candidates and breadth-first-search (BFS) has been shown to minimize
both cuckoo-path length and critical section
time~\cite{memc3,cuckoo-improvements}.
%While insertions require large
%critical sections to perform search and execute updates along the
%cuckoo path
Farm~\cite{farm} and Reno~\cite{reno}, two systems based on hopscotch
hashing, completely avoid executing long hopscotch chains due to their
execution time and complexity.  Moreover, under either approach, the
insert operation can fail despite vacant entries in the table---they
are just too far away to be reached by either the cuckoo path or
hopscotch's neighborhood-bounded linear probing.  The point at which
inserts being to fail, known as the \emph{maximum fill factor}, is a
function of the number of hash locations and row associativity in
cuckoo hashing and desired neighborhood size for hopscotch hashing.

RCuckoo uses cuckoo rather than hopscotch hashing due to locking
concerns.
%While both alog
%also limits a key's location to a bounded set of 
%bounded range of addresses and we considered using it as our core data
%structure. We believe that although hopscotch hashing can likely be
%disaggregated efficiently it is less straightforward than cuckoo
%hashing for the following reasons:
%%
First, each step of a cuckoo insert process requires one update---to
the entry being moved to its secondary location---rather than two.
When an entry is relocated in a hopscotch table, the collision bitmask
must also be updated.  (Reno~\cite{reno} uses one-sided atomics to
sloppily update the bitmask but requires a server-side CPU to fix the
bitmasks whenever concurrent inserts execute.)  Second, keys exist in
one of two locations in cuckoo hashing so updates and deletes require
locking only two rows, while hopscotch entries inhabit a range of
locations, so a conservative locking strategy must lock the entire
range.  That said, RCuckoo takes inspiration from hopscotch neighborhoods and employs dependent hashing to increase the spatial
locality of keys' potential locations, enabling clients to use local
caches to speculatively compute cuckoo paths.

%Because each lock
%acquisition is expensive this increases the fundamental difficulty of
%disaggregation a hopscotch hash.



%%

%Cuckoo hashes use
%associative rows to improve their maximum fill factors.

%%now I want to explain why cuckoo hashing and disaggregation don't work well together.

%% Lockless $O(1)$ reads make Cuckoo hashing a desireable
%% candidate for a disaggregated index. However, long
%% unpredictable cuckoo paths and RDMA CAS limitations make
%% performing insertions without locks difficult in the
%% disaggregated setting. We designed an RDMA based cuckoo hash
%% to illustrate the difficulties of performing opportunistic
%% insertions. On inserts this system makes a sequence of reads
%% to calculate a valid cuckoo path and then itteratitivly
%% issues CAS operations to swap value along the path. If any
%% value on the path is concurrently modified by another client
%% the insertions will fail and must restart.
%% Figure~\ref{fig:cuckoo-problems}(a) shows how the failure
%% rate of insertions filling a table from 80-90\% full.


%% \subsection{Full disaggregation}

%% %%
%% Disaggregated key-value in contrast assume that a memory
%% server cannot provide serialization and orchestrate their
%% writes solely using
%% clients~\cite{rolex,fusee,clover,sherman,ford,race}. With
%% the exception of Sherman~\cite{sherman} these systems use
%% systems commit writes optimistically using 64 bit RDMA CAS. 
%% %%
%% Opportunistic writes have the advantage that updates are
%% atomically visible, no critical sections exist, and client
%% failures do not leave the table in an inconsistent state.
%% Unfortunately CAS based opportunistic writes perform poorly
%% under contention leading to high tail latencies
%% ~\cite{clover}. Additionally RDMA CAS does not scale
%% well~\cite{design-guidelines}(Figure~\ref{fig:rdma-benchmarks}(b)),
%% and their small word width and lack of multi-CAS support
%% constrain the size and types of updates they can perform.

%% CAS width in particular effects system design because
%% key-value pairs can rarely fit into 64 bits, indexes updated
%% with CAS must reference keys values indirectly with a
%% pointer. At minimum resolving a remote pointer requires an
%% additional round trip for every read~\cite{race,clover}.
%% %%
%% As we will show in the following section data structures
%% like cuckoo and hopscotch hashes are difficult implement with
%% optimistic CAS updates because they require multiple updates
%% to execute atomically.
