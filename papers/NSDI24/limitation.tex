\section{Limitations}
\label{sec:limations}

\subsection{replication and persistance}
At the time of writing Rcuckoo does not support multiple
memory nodes for replication and is not designed for
persistance. No aspect of RCuckoo's algorithm prevents the
use of multiple replicas. As Rcuckoo uses lock based
protection, replicated updates can be broadcast by writes to
each replica prior to releasing locks. Rcuckoo's index is
not designed for persistance. As Rcuckoo is primarily an
index, we see it as compatable with the persistance
strategies of other disaggregated projects. For instance
RCuckoo could easily make use of Clover's persistent extent
algorithm with no changes to RCuckoo's index structure.

\subsection{Client Scalability} RCuckoo's algorithm relies
on the semantics of in order delivery for the correctness of
it's locking and reading algorithms. RDMA NIC's have hard
caps on the number of reliable connections they support
~\todo{(~65k)} these limitations are fundamental to the
cache size on the RDMA NIC~\cite{erpc,faast}, although cache
sizes have steadily increased~\cite{storm}. RDMA unreliable
connections currently do not support one-sided verbs making
them unusable for disaggregated applications.
