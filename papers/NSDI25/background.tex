
\section{Background}
\label{sec:background}

Disaggregated Key-value stores are the highest performance technology for sharing disaggregated
memory with a lineage in RDMA key-value stores. In this section we will overview that lineage and
discuss the algorithmic distinctions between the two. Our hash table design draws strongly from on
Cuckoo and hopscotch hashing. We provide a brief overview of both hash table designs and related
literature.


% Key/value stores have long used RDMA to accelerate
% performance~\cite{farm,memc3,erpc,herd,faast,mica,pilaf,cell,storm}.  These systems strike a careful
% balance between directly accessing remote memory with one-sided RDMA requests and serializing
% mutating operations at a server-side CPU.  Cuckoo~\cite{cuckoo} and hopscotch~\cite{hopscotch}
% hashing are popular choices in this space because clients can calculate the location of keys without
% needing to consult the server and potentially perform lock-less reads with
% RDMA~\cite{farm,pilaf,cuckoo}.

\subsection{Disaggregated Key-Value Stores}

The primary difference between disaggregated KVS and prior RDMA KVS is that the former assumes that
servers do not have a CPU capable of serializing requests near memory. Precursor RDMA KV stores
achieved performance by striking a delicate balance between CPU-bypass one-sided RDMA operations and
CPU-serialized two-sided operations~\cite{farm,herd,pilaf,cell,storm}. Disaggregated KVS must only
use one-sided operations even when performing complex operations like resolving a collision while
inserting.

In general the performance of a disaggregated KVS hinges write serialization performance.  Fast
consistent writes with one-sided RDMA are hard because RDMA atomics are small (8-bytes) and the cost
of round trips is high. This has lead to a divide in the design between KVS that have ordered keys
for range queries and those than do not. Those with ordered keys use locks~\cite{sherman,smart} for
their complex updates, and those with unordered keys use lockless optimistic approaches and cram
their atomic updates into 8-byte writes. These latter designs have hash indexes with 8-byte entries
which point to uncontested extent regions with the entries actual data. Index entries typically use
48 of the 64 bits as a pointer and the remaining 16bits for a digest of the key, and the size of the
data.  This leads to additional round trips to check for a keys presence or limitations in the size
of values an entry can support~\cite{fusee,race}.

KVS with B-Trees or Radix Trees for ordered keys~\cite{sherman,smart} use locks because their
updates are complex.  The advantage of locks is that these structures support inline index updates
and thus faster reads at the cost of expensive writes. Lock performance is dominated by lock
granularity and critical section size. At the time of writing all lock based KVS's assume that
clients are grouped into colocated servers which can locally candidate lock accesses and batch their
writes~\cite{sherman,smart}, an assumption which is not true for optimistic
KVS~\cite{clover,fusee,race,ditto}. Our stance is that the the performance gain of inlined
operations can make locks practical without assuming co-location if critical sections are kept to
1-RTT and locks are as fine grained as possible.

\subsection{RDMA and Network Performance} Historically network bandwidth has limited KVS store
throughput more than operation rate. For instance a ConnectX3 NIC at 40GBPs can process 75 million
packets per second would bottleneck at only 5MOPS making 1K reads. Similarly at these lower packet
rates the effects of contention on skewed distributions are not as noticeable. As bandwidths get
higher (400GBPs+) latency and contention, rather than bandwidth will become the primary scalability
bottleneck. To this end we focus on small key workloads which are limited by round trip access
latency and contention and not bandwidth. We make specific design choices which reduce round trips
at the cost of bandwidth.

Figure~\ref{fig:rdma-benchmarks}(a) shows the tradeoff in round trip time for RDMA reads of
increasing size. Critically the round trip time of a 1KB packet is lower than the latency of issuing
two dependent small packets.  The implication is that performance can be gained if a single large
packet can complete the work of two dependent smaller messages. Figure~\ref{fig:rdma-benchmarks}(b)
shows the performance overheads of using an extent based design on small key-value pairs. Inlined
values are read directly from the index while all other values are read from indirect storage. Here
client caching is turned on and clients optimistically read indirect storage in parallel with their
index read. Only if a value has changed do clients re-read from extent storage. Due to atomic width
limitations existing KV stores can not gain this performance.


% \subsection{RDMA}
%%
% RDMA allows clients to directly access the memory of a remote server with one-sided operations like
% read, write and atomic update without the involvement of a remote CPU~\cite{infiniband-spec}.  To
% set the context for RCuckoo's design, we benchmark the baseline performance of our testbed hardware,
% illustrating the constraints and opportunities.

% Atomic operations such as compare-and-swap (CAS) are essential for implementing locks or
% opportunistic concurrency. Atomics are limited to 64-bit operations and bottleneck at lower rates
% than reads and writes because they block requests on data-dependent addresses while waiting on PCIe
% round trips~\cite{design-guidelines,sherman}.  Figure~\ref{fig:rdma-benchmarks}(a) shows that the
% NICs in our testbed (100-Gbps NVIDIA Mellanox ConnectX-5s) are capable of serving many 10s of
% millions of RDMA operations per second (limited only by link speed), but CAS operations to remote
% server memory top out around 50 MOPS.  Hence, we employ atomic operations judiciously in RCuckoo.


% While atomic operations are limited to 64 bits, read and write message sizes are bounded only by the
% link MTU.  Figure~\ref{fig:rdma-benchmarks}(b) shows that on our testbed NIC-to-NIC round-trip times
% are similar for all message sizes less than about 128 bytes, and messages must exceed 1~KB before
% the latency of a single large operation exceeds two RTTs of smaller ones.  We leverage this
% observation in RCuckoo by collapsing multiple small reads into a single larger one when appropriate.
% The optimal threshold trades off latency reduction against read amplificaiton.


% Finally, we highlight that our testbed NICs, like those from several
% vendors, include a small amount (256~KB in our case) of on-NIC memory
% that can be addressed by remote clients using RDMA
% operations~\cite{device-memory}.  Accesses to NIC memory avoid the
% need to cross the server's PCIe bus, decreasing latency and increasing
% throughput.  The performance gain is particularly significant for
% atomic operations.  Figure~\ref{fig:rdma-benchmarks}(c) shows the
% maximum aggregate throughput of concurrent CAS operations targeting
% the same single (i.e., contended) address or distinct, independent
% addresses in both main server memory (shown in orange) and on-NIC
% device memory (blue).  CAS operations perform between 1.8--3.1$\times$
% faster on NIC memory.  RCuckoo's datastructures are designed so that
% all CAS operations are performed on NIC memory.


\subsection{Concurrent hash tables} 
\label{sec:cuckoo-back}

Fully disaggregated key/value stores are essentially concurrent hash
tables whose conflict resolution strategy is implemented entirely by
individual clients~\cite{rolex,fusee,race}.  Like any hash table, the
underlying hashing algorithm must have an approach to managing
collisions. Cuckoo and hopscotch hashing are particularly attractive
in this context, because they both provide the property that the
potential locations of an entry in the table, regardless of
contention or collision, can be deterministically computed by clients
based only upon the key itself.  Moreover, the set of locations is
limited.  Hence, at least in theory, systems built around either
cuckoo or hopscotch hashing hold the potential for single-round-trip
reads.

%\todo{Make sure to describe how a cuckoo path works}

\textbf{Cuckoo hashing} uses independent hash functions to compute two
potential table locations for a key, a primary and a secondary, where
each location corresponds to an associative row of entries.  A key is
always inserted into its primary location.  If that row is full, an
existing key is evicted (or ``cuckooed'') to its secondary row to make
space. If the cuckooed entry's secondary row is also full, the process
iterates (by selecting yet another entry in the secondary row to
cuckoo) until an open location is found. The path of evictions is
known as a~\textit{cuckoo path}.  While insertions can be involved,
reads can always be executed in a single round trip by reading the
rows corresponding to both of a key's locations
simultaneously~\cite{pilaf}.

\textbf{Hopscotch hashing} works in a similar fashion but provides a
slightly different guarantee, namely that keys will be located within
a bounded neighborhood.  (While cuckoo hashing limits the number of
locations in which a key may be stored, it does not provide any
locality guarantees regarding those locations.) It does so by finding
the physically closest empty entry to the desired location and then,
if that location is not within the neighborhood, iteratively
moving other entries out of the way to make room for the new key.
The hopscotch process is facilitated by maintaining a per-entry
bitmask of nearby collisions.  As with cuckoo hashing, clients can read in a single round trip time by reading a key's entire
neighborhood at once.

The insert operation is expensive for both approaches, and prior
systems have taken steps to mitigate its cost.  In associative hashes
like cuckoo hash tables, multiple entries can be chosen as eviction
candidates and breadth-first-search (BFS) has been shown to minimize
both cuckoo-path length and critical section
time~\cite{memc3,cuckoo-improvements}.
%While insertions require large
%critical sections to perform search and execute updates along the
%cuckoo path
Farm~\cite{farm} and Reno~\cite{reno}, two systems based on hopscotch
hashing, completely avoid executing long hopscotch chains due to their
execution time and complexity.  Moreover, under either approach, the
insert operation can fail despite vacant entries in the table---they
are just too far away to be reached by either the cuckoo path or
hopscotch's neighborhood-bounded linear probing.  The point at which
inserts being to fail, known as the \emph{maximum fill factor}, is a
function of the number of hash locations and row associativity in
cuckoo hashing and desired neighborhood size for hopscotch hashing.

RCuckoo uses cuckoo rather than hopscotch hashing due to locking
concerns.
%While both alog
%also limits a key's location to a bounded set of 
%bounded range of addresses and we considered using it as our core data
%structure. We believe that although hopscotch hashing can likely be
%disaggregated efficiently it is less straightforward than cuckoo
%hashing for the following reasons:
%%
First, each step of a cuckoo insert process requires one update---to
the entry being moved to its secondary location---rather than two.
When an entry is relocated in a hopscotch table, the collision bitmask
must also be updated.  (Reno~\cite{reno} uses one-sided atomics to
sloppily update the bitmask but requires a server-side CPU to fix the
bitmasks whenever concurrent inserts execute.)  Second, keys exist in
one of two locations in cuckoo hashing so updates and deletes require
locking only two rows, while hopscotch entries inhabit a range of
locations, so a conservative locking strategy must lock the entire
range.  Yet, RCuckoo takes inspiration from hopscotch neighborhoods and employs dependent hashing to increase the spatial
locality of key locations, enabling clients to use local
caches to speculatively compute cuckoo paths.

%Because each lock
%acquisition is expensive this increases the fundamental difficulty of
%disaggregation a hopscotch hash.



%%

%Cuckoo hashes use
%associative rows to improve their maximum fill factors.

%%now I want to explain why cuckoo hashing and disaggregation don't work well together.

%% Lockless $O(1)$ reads make Cuckoo hashing a desireable
%% candidate for a disaggregated index. However, long
%% unpredictable cuckoo paths and RDMA CAS limitations make
%% performing insertions without locks difficult in the
%% disaggregated setting. We designed an RDMA based cuckoo hash
%% to illustrate the difficulties of performing opportunistic
%% insertions. On inserts this system makes a sequence of reads
%% to calculate a valid cuckoo path and then itteratitivly
%% issues CAS operations to swap value along the path. If any
%% value on the path is concurrently modified by another client
%% the insertions will fail and must restart.
%% Figure~\ref{fig:cuckoo-problems}(a) shows how the failure
%% rate of insertions filling a table from 80-90\% full.


%% \subsection{Full disaggregation}

%% %%
%% Disaggregated key-value in contrast assume that a memory
%% server cannot provide serialization and orchestrate their
%% writes solely using
%% clients~\cite{rolex,fusee,clover,sherman,ford,race}. With
%% the exception of Sherman~\cite{sherman} these systems use
%% systems commit writes optimistically using 64 bit RDMA CAS. 
%% %%
%% Opportunistic writes have the advantage that updates are
%% atomically visible, no critical sections exist, and client
%% failures do not leave the table in an inconsistent state.
%% Unfortunately CAS based opportunistic writes perform poorly
%% under contention leading to high tail latencies
%% ~\cite{clover}. Additionally RDMA CAS does not scale
%% well~\cite{design-guidelines}(Figure~\ref{fig:rdma-benchmarks}(b)),
%% and their small word width and lack of multi-CAS support
%% constrain the size and types of updates they can perform.

%% CAS width in particular effects system design because
%% key-value pairs can rarely fit into 64 bits, indexes updated
%% with CAS must reference keys values indirectly with a
%% pointer. At minimum resolving a remote pointer requires an
%% additional round trip for every read~\cite{race,clover}.
%% %%
%% As we will show in the following section data structures
%% like cuckoo and hopscotch hashes are difficult implement with
%% optimistic CAS updates because they require multiple updates
%% to execute atomically.
