\section{Limitations}
\label{sec:limations}

\textbf{Fault Tolerance:} At the time of
writing Rcuckoo does not support multiple memory nodes for
replication and is not designed for persistance. No aspect
of RCuckoo's algorithm prevents the use of multiple
replicas. Because Rcuckoo uses lock based protection,
replicated updates can be broadcast by writes to each
replica prior to releasing locks. Rcuckoo's index is not
designed for persistance. As Rcuckoo is primarily an index,
we see it as compatible with the persistance strategies of
other disaggregated projects~\cite{clover}. For instance
RCuckoo could easily make use of Clover's persistent extent
algorithm with no changes to RCuckoo's index structure.

\textbf{Memory Scalability:} Currently RCuckoo's memory index
is limited to a single machine. Multiple memory machines
could be incorporated by partitioning the table across
machines and having each memory machine maintain a lock
table for each partition. The would require some updates to
the locking algorithm as currently lock and unlock use a
single reliable connection which guarantees in order
delivery. Therefore inserts, updates and deletes spanning
multiple machines would require additional delays to ensure
in order locking.

\textbf{Client Scalability:} RCuckoo's algorithm relies on
the semantics of in order delivery for the correctness of
it's locking and reading algorithms. RDMA NIC's have hard
caps on the number of reliable connections they support
before inducing cache thrashing. these limitations are
fundamental to the cache size on the RDMA
NIC~\cite{erpc,faast,rma}, although cache sizes have
steadily increased~\cite{storm}. RDMA unreliable connections
currently do not support one-sided verbs making them
unusable for disaggregated applications.

\textbf{Range Queries:} Our currently implementation does
not support efficient range queries like other ordered
key-value stores~\cite{sherman,rolex}. Dependent hashing
spreads keys randomly throughout the table relative to one
another. We don't consider RCuckoo's index structure to be
compatible with range queries.

\section{Future Work}
\label{sec:future}

\textbf{Other Index Structures:} Our locality based hashing
algorithm may be applicable for other data structures.
Hopscotch hashing~\cite{hopscotch} is another hash index
with similarities to the design decisions we made in
RCuckoo. Hopscotch hashing has similar goals to RCuckoo in
reducing read latency, and has similar issues with locks
being scattered a long distance from initial insert
locations. Using techniques to bound hopscotch distances
could yield similar performance benefits to RCuckoo.

\textbf{Generalized Data Structures:} Disaggregated data
structures are largely based on opportunistic concurrency.
Some structures like hash tables are relatively easy to
apply these techniques to, while others such as binary trees
are hard~\cite{nonb-binary}. In future work we hope to
explore how to use or locking techniques on pointer based
remote data structures.
