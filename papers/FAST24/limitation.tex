\section{Limitations}
\label{sec:limations}

\subsection{replication and persistance} At the time of
writing Rcuckoo does not support multiple memory nodes for
replication and is not designed for persistance. No aspect
of RCuckoo's algorithm prevents the use of multiple
replicas. Because Rcuckoo uses lock based protection,
replicated updates can be broadcast by writes to each
replica prior to releasing locks. Rcuckoo's index is not
designed for persistance. As Rcuckoo is primarily an index,
we see it as compatible with the persistance strategies of
other disaggregated projects~\cite{clover}. For instance
RCuckoo could easily make use of Clover's persistent extent
algorithm with no changes to RCuckoo's index structure.

\subsection{scalability} 
\textbf{Memory Scalability} Currently RCuckoo's memory index
is limited to a single machine. Multiple memory machines
could be incorporated by partitioning the table across
machines and having each memory machine maintain a lock
table for each partition. The would require some updates to
the locking algorithm as currently lock and unlock use a
single reliable connection which guarantees in order
delivery. Therefore inserts, updates and deletes spanning
multiple machines would require additional delays to ensure
in order locking.

\textbf{Client Scalability} RCuckoo's algorithm relies
on the semantics of in order delivery for the correctness of
it's locking and reading algorithms. RDMA NIC's have hard
caps on the number of reliable connections they support
~\todo{(~65k)} these limitations are fundamental to the
cache size on the RDMA NIC~\cite{erpc,faast}, although cache
sizes have steadily increased~\cite{storm}. RDMA unreliable
connections currently do not support one-sided verbs making
them unusable for disaggregated applications.
