\section{Introduction}
\label{sec:intro}


DRAM has become a scarce resource. As CPU core counts
continue to rise, per-core capacity has dropped in the past
decade. In response data center architects have proposed
resource disaggregation to improve resource scalability, and
utilization. In the disaggregated model, resources (compute,
memory, storage) are physically distant, and interconnected
by a high speed network. In a fully disaggregated model
distinct servers house resources such as memory servers
which contain only memory, and no colocated CPU for
processing.

% DRAM capacity is scaling at an ever decreasing rate. As such
% the demand in data centers for high efficiency memory
% systems grows every year to improve memory utilization and
% sharing. Memory disaggregation i.e separating compute and
% memory into large pools is a promising approach to increase
% a CPU's per core memory access beyond what is available in a
% single machine while simultaneously decreasing per machine
% memory stranding.

Full disaggregation is challenging as it requires new system
designs to cope with the additional network latency.  For
example accessing intra-rack remote DRAM has a 20x (50ns to
1us) overhead in relation to local memory. Cores sharing
remote memory must suffer a proportionally high cost for
synchronization when sharing pools of memory. Many prior
disaggregated systems have eschewed sharing for static
partitioning to avoid the high overheads~\cite{legoos,
blade-server, reigons}.  Some projects have opted
for a partially disaggregated model, in which a CPU core is
colocated with memory to provide
serialization~\cite{clover}, sharing a close design with
RDMA key-value stores optimized with one-sided RDMA
operations ~\cite{herd,erpc,pilaf,cell,clover}. While highly
performant these approaches do not conform to fully
disaggregated designs.

% to remote memory are approximately 20x that
% of local DRAM (50ns to 1us) which severely increases the
% cost of synchronizing shared data structures among many
% CPU's across multiple machines. Traditional and partially
% disaggregated key-value serialize updates with a CPU
% colocated with
% memory~\cite{memcached,herd,erpc,pilaf,cell,clover}(\todo{check
% sherman}). This approach does not meet the requirements of
% full disaggregation in which memory is a passive resource.
% In the fully disaggregated setting serialization is achieved
% entirely with a client based protocol. 

Some prior work has examined the design of fully
disaggregated hash table indexes~\cite{sherman, race, fusee,
rolex}.  These approaches either use optimistic concurrency,
or assume that most clients are colocated and can resolve
their lock acquisitions in local memory. Optimistic
approaches using RDMA atomics are limited by the width of
the atomic instruction (64 bits for compare and swap). 64
bits is often too small to encapsulate a full key-value
update. The result is that optimistic approaches require
more round trips to remote memory to complete their
operations. These approaches have dismissed lock based
approaches outright as the size of their critical sections
can bottleneck performance, and  the cost of acquiring and
releasing locks (at minimum two round trips) is often higher
than the optimistic approach (two or three round trips).

% %concurrent data structures round trips
% Concurrent data structure design for remote memory is hard.
% Access latency to remote memory is high so round trips per
% operation must be minimized to achieve efficiency.
% Serialization is particularly hard because there is no
% centralized serialization point to guard access to remote
% memory. RDMA NIC's provide atomic verbs such as
% compare-and-swap (CAS), but these are by no means a silver
% bullet.  Each atomic request takes a round trip from client
% to server to execute. In the best case lock/unlock requires
% two round trips, if multiple locks are required, or locks
% are contested the number of round trips increases.

% %cpu locking vs rdma
% In traditional key-value stores (Memcached~\cite{memcached})
% the CPU coordinates table access for read, write, and lock
% instructions. In contrast RDMA based Key-value
% stores~\cite{herd,erpc,pilaf} use a mixture of one-sided (no
% cpu) and two-sided (cpu involved) verbs to alleviate the CPU
% bottleneck. Reads are typically one sided to bypass the CPU
% bottleneck~\cite{pilaf,cell} while writes are typically two
% sided so memory-side CPU can orchestrate serialized
% operations (e.g. locking) with main memory access latency
% (50-100ns).  These small access times keep critical sections
% small for CPU based locking, and dramatically increase them
% for one-sided RDMA based locking schemes~\cite{clover,
% sherman}.

%rdma and cuckoo hashing
In this work we examine the tradeoffs between optimistic and
lock based data structures for full disaggregation. Our
contribution is a new concurrent hash table design which
uses a novel cuckoo hashing algorithm. Our algorithm
(rcuckoo) is designed to enable locking and unlocking
efficiently using one-sided RDMA. In the common case our
reads execute in a single round trip, and updates to the
table are performed in two round trips. 

Our algorithm makes use of multiple novel contributions
which decrease round trips per operation. The first is a
dependent hashing algorithm. Dependent hashing increases the
likelihood that a key's hash locations will be spatially
near one another. We find that locality increases the
throughput of each table operation (insert, read). Inserting
into a cuckoo hash table normally requires acquiring many
locks randomly throughout the table, dependent hashing
clusters lock acquisitions enabling many locks to be
acquired in a single atomic request. Second, our algorithm
uses speculative lock acquisition. Locks are acquired in
bulk based on a local cache of the remote index. Once locks
are held a second search on synchronized locked data is
performed to ensure a correct insertion. Third we use
locality to improve both read and search speeds. We find
that using a single large read over two smaller reads
improves throughput when key locations are known to be
physically close.

\sg{Sherman uses masked compare and swap as well as device
memory, we might use it better but it's hard to claim a
contribution here.}

We develop a prototype of our algorithm and evaluate it
against two recent RDMA based key value stores. Our
evaluation shows that we can achieve ~\todo{2x faster reads
and 1.5x faster writes on a multitude of ycsb workloads.}